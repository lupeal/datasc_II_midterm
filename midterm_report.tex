% Options for packages loaded elsewhere
\PassOptionsToPackage{unicode}{hyperref}
\PassOptionsToPackage{hyphens}{url}
%
\documentclass[
]{article}
\usepackage{amsmath,amssymb}
\usepackage{iftex}
\ifPDFTeX
  \usepackage[T1]{fontenc}
  \usepackage[utf8]{inputenc}
  \usepackage{textcomp} % provide euro and other symbols
\else % if luatex or xetex
  \usepackage{unicode-math} % this also loads fontspec
  \defaultfontfeatures{Scale=MatchLowercase}
  \defaultfontfeatures[\rmfamily]{Ligatures=TeX,Scale=1}
\fi
\usepackage{lmodern}
\ifPDFTeX\else
  % xetex/luatex font selection
\fi
% Use upquote if available, for straight quotes in verbatim environments
\IfFileExists{upquote.sty}{\usepackage{upquote}}{}
\IfFileExists{microtype.sty}{% use microtype if available
  \usepackage[]{microtype}
  \UseMicrotypeSet[protrusion]{basicmath} % disable protrusion for tt fonts
}{}
\makeatletter
\@ifundefined{KOMAClassName}{% if non-KOMA class
  \IfFileExists{parskip.sty}{%
    \usepackage{parskip}
  }{% else
    \setlength{\parindent}{0pt}
    \setlength{\parskip}{6pt plus 2pt minus 1pt}}
}{% if KOMA class
  \KOMAoptions{parskip=half}}
\makeatother
\usepackage{xcolor}
\usepackage[margin=1in]{geometry}
\usepackage{longtable,booktabs,array}
\usepackage{calc} % for calculating minipage widths
% Correct order of tables after \paragraph or \subparagraph
\usepackage{etoolbox}
\makeatletter
\patchcmd\longtable{\par}{\if@noskipsec\mbox{}\fi\par}{}{}
\makeatother
% Allow footnotes in longtable head/foot
\IfFileExists{footnotehyper.sty}{\usepackage{footnotehyper}}{\usepackage{footnote}}
\makesavenoteenv{longtable}
\usepackage{graphicx}
\makeatletter
\def\maxwidth{\ifdim\Gin@nat@width>\linewidth\linewidth\else\Gin@nat@width\fi}
\def\maxheight{\ifdim\Gin@nat@height>\textheight\textheight\else\Gin@nat@height\fi}
\makeatother
% Scale images if necessary, so that they will not overflow the page
% margins by default, and it is still possible to overwrite the defaults
% using explicit options in \includegraphics[width, height, ...]{}
\setkeys{Gin}{width=\maxwidth,height=\maxheight,keepaspectratio}
% Set default figure placement to htbp
\makeatletter
\def\fps@figure{htbp}
\makeatother
\setlength{\emergencystretch}{3em} % prevent overfull lines
\providecommand{\tightlist}{%
  \setlength{\itemsep}{0pt}\setlength{\parskip}{0pt}}
\setcounter{secnumdepth}{-\maxdimen} % remove section numbering
\usepackage{setspace}\doublespacing
\captionsetup[figure]{font=8}
\ifLuaTeX
  \usepackage{selnolig}  % disable illegal ligatures
\fi
\IfFileExists{bookmark.sty}{\usepackage{bookmark}}{\usepackage{hyperref}}
\IfFileExists{xurl.sty}{\usepackage{xurl}}{} % add URL line breaks if available
\urlstyle{same}
\hypersetup{
  pdftitle={P8106 Midterm Project: Predicting COVID-19 Recovery Time},
  pdfauthor={Guadalupe Antonio Lopez, Gustavo Garcia-Franceschini, Derek Lamb; UNI's: GA2612, GEG2145, DRL2168},
  hidelinks,
  pdfcreator={LaTeX via pandoc}}

\title{P8106 Midterm Project: Predicting COVID-19 Recovery Time}
\author{Guadalupe Antonio Lopez, Gustavo Garcia-Franceschini, Derek
Lamb \and UNI's: GA2612, GEG2145, DRL2168}
\date{}

\begin{document}
\maketitle

\hypertarget{introduction}{%
\subsection{Introduction}\label{introduction}}

This analysis combines three cohort studies regarding recovery time from
COVID-19 illness. We have the individual's gender and race, along with
other medical information. Among these, stand out their vaccination
status and the study (A or B) they were a part of. With this
information, we aim to fit a model that can both help us predict
recovery time, and understand variables strongly associated with
increased risk for long COVID-19 recovery times.

\hypertarget{exploratory-data-analysis}{%
\subsection{Exploratory Data Analysis}\label{exploratory-data-analysis}}

To start our investigation, we conduct exploratory data analysis. To
explore our data and train the models, we partitioned the data into
training and testing sets, with 80\% of the data (2400 subjects) being
assigned to the training set, and the remaining 20\% (600 subjects)
being assigned to the test set. This way, the test set is not included
in our EDA. The split was done with a random seed of 1.

We have 15 variables, including our response (recovery time). We
calculated summary statistics for 14 of them, grouping them by study
group, since we are interested in knowing whether the two study groups
are similar (\textbf{Table 1}). We noticed that they are very similar in
all our proposed covariates, yet the mean recovery time for individuals
in gorup B is five days more than for individuals in group A. It is also
important to note that group A has almost twice as many individuals as
group B.

\begin{longtable}[]{@{}lcc@{}}
\caption{Summary statistics}\tabularnewline
\toprule\noalign{}
\textbf{Variable} & \textbf{A}, N = 1,598 & \textbf{B}, N = 802 \\
\midrule\noalign{}
\endfirsthead
\toprule\noalign{}
\textbf{Variable} & \textbf{A}, N = 1,598 & \textbf{B}, N = 802 \\
\midrule\noalign{}
\endhead
\bottomrule\noalign{}
\endlastfoot
\textbf{age} & 60.252 (4.581) & 60.152 (4.401) \\
\textbf{gender} & 769 (48\%) & 394 (49\%) \\
\textbf{race} & & \\
1 & 1,039 (65\%) & 511 (64\%) \\
2 & 88 (5.5\%) & 43 (5.4\%) \\
3 & 331 (21\%) & 166 (21\%) \\
4 & 140 (8.8\%) & 82 (10\%) \\
\textbf{smoking} & & \\
0 & 966 (60\%) & 481 (60\%) \\
1 & 448 (28\%) & 237 (30\%) \\
2 & 184 (12\%) & 84 (10\%) \\
\textbf{height} & 169.827 (5.919) & 169.937 (6.005) \\
\textbf{weight} & 79.933 (7.182) & 80.211 (7.301) \\
\textbf{bmi} & 27.775 (2.786) & 27.842 (2.875) \\
\textbf{hypertension} & 799 (50\%) & 388 (48\%) \\
\textbf{diabetes} & 266 (17\%) & 118 (15\%) \\
\textbf{SBP} & 130.594 (8.119) & 130.203 (7.787) \\
\textbf{LDL} & 110.379 (19.754) & 110.642 (20.294) \\
\textbf{vaccine} & 964 (60\%) & 474 (59\%) \\
\textbf{severity} & 176 (11\%) & 85 (11\%) \\
\textbf{recovery\_time} & 40.620 (11.215) & 45.727 (37.982) \\
\end{longtable}

We further investigate the relationship between study group and recovery
time in \textbf{Figure 1}. We found that the COVID-19 infection recovery
time is heavily right-skewed, regardless of the study group. However,
Study A has a later peak, while Study B has a heavier tail,
corresponding to more individuals in that study experiencing longer
recovery time. This is an early indication that study group might be an
important variable when predicting recovery time.

\begin{figure}

{\centering \includegraphics[width=0.65\linewidth]{midterm_report_files/figure-latex/density_plot_by_study-1} 

}

\caption{Recovery time density, by study group}\label{fig:density_plot_by_study}
\end{figure}

In \textbf{Figure 2}, we plotted our continuous variables on the x-axis,
with recovery time on the y-axis. We see that bmi and height have some
non-linearity, with higher bmi and lower height associated with higher
recovery times, but values in the middle or opposing end not showing any
particular pattern. Age, weight, SBP and LDL show some outliers in the
middle of their ranges. These observations suggest we should implement
models that account for non-linearity.

\begin{figure}

{\centering \includegraphics[width=0.65\linewidth]{midterm_report_files/figure-latex/scatterplot-1} 

}

\caption{Continuous variables plotted against recovery time}\label{fig:scatterplot}
\end{figure}

We also examined the pairwise correlations of the variables, and the
correlations of the covariates with the recovery time. There were two
clusters of strong correlation (height, weight, and BMI; hypertension
and SBP), but these covariates were functionally dependent upon each
other. There were no other strong correlations between variables, and no
one covariate had an exceptional correlation to recovery time.

\begin{figure}

{\centering \includegraphics[width=0.65\linewidth]{midterm_report_files/figure-latex/corrplot-1} 

}

\caption{Variable correlation plot}\label{fig:corrplot}
\end{figure}

\hypertarget{model-training}{%
\subsection{Model Training}\label{model-training}}

To predict COVID-19 recovery time, we modeled the data using four
approaches -- two linear and two non-linear. For the linear approaches,
we selected elastic net and partial least squares regression. For the
nonlinear approaches, we selected multivariate adaptive regression
splines (MARS) and a general additive model (GAM). As specified before,
we use our training set to train all models.

All models were fitted using the \texttt{train()} function in the
\texttt{caret} package. Although some inputs varied by model, the common
inputs were formula or model matrix and response vector, data, method,
tuning parameters grid, and a 10-fold cross validation method.
\textbf{Note that all models were fit using a seed of 1.}

\hypertarget{elastic-net-model}{%
\subsubsection{Elastic Net Model}\label{elastic-net-model}}

To fit the elastic net model, we used the model formula with
\texttt{recovery\_time} as the response and all other variables in our
training data set to be predictors. Given that we fit an elastic net
model, the method specified was \texttt{glmnet}, with tuning parameter
alpha to be sequenced between 0 and 1 (with length 21) and lambda to be
exponentially sequenced between -6 and 1 (with length 100). We settled
on this lambda region after fitting the model various times with
different regions. We started with a large region (-4 to 4), but
realized that our preferred lambda value was close to our lower
boundary. Thus, we continued to expand our region until we settled on -6
to 1.

After fitting the elastic net model, the final model based on the
optimal lambda contained 17 predictors and an intercept -- no predictors
were shrunk to zero. The values for each predictor represent the
estimated effect of each predictor on recovery time. Based on our
output, age, height, bmi, and systolic blood pressure had positive
coefficients. This suggests that an increase in a given predictor is
associated with an increase in recovery time. There were also positive
coefficients for categorical variables race (asian), former and current
smoking status, hypertension, systolic blood pressure, severity, and
study B. These positive coefficients indicate the difference in the
outcome compared to their reference level.

\hypertarget{partial-least-squares-pls-regression-model}{%
\subsubsection{Partial Least Squares (PLS) Regression
Model}\label{partial-least-squares-pls-regression-model}}

To fit the PLS model, we used the training model matrix, based on our
training data, and training response vector. Given that we fit a PLS
model, the method specified was \texttt{pls}, with number of components
ranging between 1 and 15. This range is based on the number of variables
in our training model matrix. Additionally, the predictor data was
centered and scaled.

After fitting the PLS model, the final model is based on 13 components.
The positive predictor coefficients in the elastic net model were also
positive in the PLS model. In this case, the positive coefficients
suggest that an increase in a predictor variable (by one standard
deviation) is associated with an increase in standardized recovery time
by the coefficient value (with standard deviation units).

\hypertarget{multivariate-adaptive-regression-splines-model}{%
\subsubsection{Multivariate Adaptive Regression Splines
Model}\label{multivariate-adaptive-regression-splines-model}}

To fit the MARS model, we used the training model matrix and training
response vector. Given that we fit a MARS model, the method specified
was \texttt{earth}, with degrees ranging between 1 and 3 and the maximum
number of terms in the pruned model to range between 2 and 14.

After fitting the MARS model, the final model is based on 12
coefficients terms including an intercept based on 9 predictors. The
predictors that most inform recovery time are bmi, height, weight,
vaccination status, study group B, LDL, systolic blood pressure,
COVID-19 severity status, and smoking status.

\hypertarget{general-additive-model}{%
\subsubsection{General Additive Model}\label{general-additive-model}}

To fit the GAM, we used the training model matrix and training response
vector. Given that we fit a GAM, the method specified was \texttt{gam}
and a 10-fold cross validation.

The final GAM contains categorical predictors gender, race (all levels),
current smoking status, hypertension, diabetes, vaccination status,
COVID-19 severity status, study group B, and smooth terms applied to
continuous variables age, systolic blood pressure, LDL, bmi, height,
weight. The model has a total estimated degrees of freedom of 35.99,
which suggests a relatively flexible model.

\hypertarget{results}{%
\subsection{Results}\label{results}}

We then compared the four models that we fit by resampling on the
training set. In the figure below, we constructed boxplots to compare
the four different models by their resampled RMSE. The two linear models
and GAM perform about the same, though GAM was a bit better on average.
MARS noticeably outperformed the other models.

\begin{figure}

{\centering \includegraphics[width=0.65\linewidth]{midterm_report_files/figure-latex/resamples-1} 

}

\caption{Resampled RMSE for our four models}\label{fig:resamples}
\end{figure}

Once we decided on MARS as a final model, we calculated the test error
using the 20\% partition of the initial data set.

The test RMSE for the MARS model is 17.18.

\hypertarget{conclusion}{%
\subsection{Conclusion}\label{conclusion}}

In this project, our goal was to use statistical learning to gain
insight into the recovery process of people infected with COVID-19. We
fit four models to predict COVID-19 recovery time from a set of 14
covariates, two linear and two nonlinear. Our linear models achieved
similar performance in predicting recovery time, but they were outdone
by the nonlinear methods, MARS in particular. This improvement in
prediction is due to the greater flexibility of the nonlinear methods,
but comes at a trade-off of the interpretability of such models.
However, as our goal was to develop the best model for predicting
COVID-19 recovery time, we are comfortable giving up some of this
interpretability, and recommending the MARS model developed above for
this task.

\end{document}
